\subsubsection{Adapter}

Gli \textbf{\textcolor{cyan}{adapter}} sono utili
quando vogliamo interagire con un'interfaccia che non
è omogenea a quella utilizzata dal \emph{client}.

Questo pattern permette di convertire l'interfaccia di una classe
in un'altra interfaccia che è compatibile con quella utilizzata dal \emph{client}.

Si usa la \emph{\textcolor{cyan}{delegazione}} per associare un interfaccia
\emph{adapter} ad una \emph{adattata}.

I \textcolor{cyan}{partecipanti} sono:
\begin{itemize}
    \item \textcolor{cyan}{Target}: l'interfaccia che il \emph{client} usa.
    \item \textcolor{cyan}{Client}: utilizza oggetti che siano conformi all'interfaccia \emph{Target}.
    \item \textcolor{cyan}{Adaptee}: definisce un'interfaccia già esistente che necessita di essere adattata.
    \item \textcolor{cyan}{Adapter}: adatta l'interfaccia \emph{Adaptee} all'interfaccia \emph{Target}.
\end{itemize}

L'\emph{Adapter} può anche ricoprire il ruolo di \emph{Concrete Strategy}
nello \emph{Strategy} pattern. In questo modo implementiamo la possibilità di cambiare
gli oggetti \emph{Adapter} a runtime.

\subsubsection{Proxy}

Questo pattern fornisce un surrogato per un altro oggetto, ovvero lo sostituisce
in modo da controllare l'accesso alle sue funzionalità. Può sembrare simile
ad \emph{\textcolor{cyan}{Adapter}} ma, a differenza di questo, il \textbf{\textcolor{cyan}{Proxy}}
può eseguire pre e post-elaborazioni, inoltre esso e l'oggetto originale hanno la stessa interfaccia.

Esistono vari tipi di \textcolor{cyan}{Proxy}:
\begin{itemize}
    \item \textcolor{cyan}{Remote Proxy}: permette l'accesso ad un oggetto remoto.
        Nello specifico il \emph{Proxy} si comporta come una rappresentazione locale di un oggetto
        che risiede in un'altra \emph{Java Virtual Machine}. Il client invoca un
        metodo del \emph{Proxy} che lo inoltra tramite la rete all'oggetto reale, una volta ricevuta la risposta, il
        \emph{Proxy} la restituisce al \emph{client}.

        Inoltre è anche possibile introdurre un altro \emph{Proxy} lato server,
        a questo punto, quello lato client lo chiameremo \emph{Client Helper}, mentre
        il proxy lato server sarà il \emph{Service Helper}. I due proxy hanno
        ovviamente la stessa interfaccia, che è la stessa dell'oggetto reale, il \emph{Service Helper}
        in particolare aggiunge un altro \emph{layer} di operazioni, spacchettando la richiesta
        del \emph{Client Helper} ed invocando i metodi opportuni dell'oggetto reale. Al contrario,
        il \emph{Service Helper} impacchetta il risultato del metodo e lo spedisce al \emph{Client Helper}
        che una volta ricevuto, lo spacchetterà e lo darà al \emph{client}.

        Nel \emph{Remote Method Invocation} di Java, il \emph{Client Helper} è chiamato
        \emph{Stub}, mentre il \emph{Service Helper} è lo \emph{Skeleton}. L'unica differenza col
        metodo precendente è che qui lato client non possediamo già lo \emph{Stub}, quindi
        il \emph{client} deve prima effettuare una richiesta di \verb|lookup()| nell'\emph{RMI Registry} al server che gli restituirà
        l'oggetto \emph{Stub} per il servizio richiesto, che potrà essere utilizzato per effettuare le richieste al server.
    \item \textcolor{cyan}{Protection Proxy}: implementa un controllo sugli accessi.
    \item \textcolor{cyan}{Cache Proxy}: mantiene coppie del tipo \verb|richiesta-risposta| sgravando
        il \emph{server} di questa responsabilità.
    \item \textcolor{cyan}{Synchronization Proxy}: gestisce gli accessi concorrenti.
    \item \textcolor{cyan}{Virtual Proxy}: si comporta come l'oggetto originale, mentre quest'ultimo viene
        costruito. Viene utilizzato principalmente quando per l'appunto l'oggetto reale
        richiede costi di creazione molto elevati. Una volta terminata la creazione, il \emph{Virtual Proxy}
        delegherà le richieste all'oggetto originale.
\end{itemize}

\subsubsection{Decorator}

Questo pattern si usa per evitare di creare un'interfaccia
che abbia un numero troppo elevato di classi che la implementano.
Una prima soluzione sarebbe quella di avere una serie di valori booleani
nella superclasse che indicano se l'oggetto presenta o no delle determinate caratteristiche, in questo modo eliminiamo un po' di sottoclassi inutili
e manteniamo solo quelle principali. Ovviamente anche questa soluzione
è errata, dato che nel caso si vuole aggiungere o rimuovere una caratteristica
si andrebbe a violare il principio \emph{Open-Closed}.

Questo pattern ci viene incontro offrendo una soluzione: l'idea è quella di trasformare
quelle caratteristiche che erano sotto forma di valori booleani, in classi, a questo punto se vogliamo
aggiungere una caratteristica al nostro oggetto basta \emph{"wrapparla"} all'interno della caratteristica che avevamo già,
e così via fino a quando non arriviamo ad una foglia, rappresentata da una delle classi che avevamo indicato come principali.

I partecipanti sono i seguenti:
\begin{itemize}
    \item \textcolor{cyan}{Component}: è l'interfaccia generale di tutti gli altri oggetti di questo pattern.
    \item \textcolor{cyan}{Concrete Component}: la classe \emph{"foglia"} di oggetti che ricevono nuove caratteristiche.
    \item \textcolor{cyan}{Decorator}: definisce un'interfaccia che estende \emph{Component}, in più mantiene un riferimento
        ad un altro oggetto di tipo \emph{Component}.
    \item \textcolor{cyan}{Concrete Decorator}: definisce una caratteristica.
\end{itemize}

Nonostante viene usata l'ereditarietà tra \emph{Decorator} e \emph{Component}, questa non viene utilizzata per aggiungere nuove
funzionalità, ma per avere un \emph{type matching} tra i \emph{Decorator} e gli oggetti
che vanno a \emph{"decorare"}. Infatti l'aggiunta di nuove funzionalità avviene non grazie
all'ereditarietà, ma per merito della composizione che ci permette di aggiungere
\emph{Decorator} a \emph{runtime}.

Questo pattern è molto utile perchè ci permette anche di attaccare ad un
oggetto la stessa caratteristica più di una volta.

Questo pattern può presentare alcuni problemi, ad esempio l'interfaccia
di un oggetto \emph{Decorator} deve essere uguale all'interfaccia dell'oggetto che và
a decorare. Inoltre se l'interfaccia \emph{Decorator} implementa una sola caratteristica, l'interfaccia
stessa và omessa e si accorpora la caratteristica direttamente in \emph{Concrete Decorator}. Dato
che la classe \emph{Component} è ereditata da componenti e decoratori è necessario
cercare di mantenerla il più leggera e semplice possibile; dato che se è complessa di conseguenza
sarà complessa anche la classe \emph{Decorator} che diventa costosa da usare in grandi
quantità. Un'altra limitazione, è che occorre usare questo pattern solo se si vuole aggiungere uno
strato all'oggetto e non si vuole modificarlo, in quel caso và usato il pattern \emph{Strategy}.

Ad esempio a differenza di \emph{Strategy} in \emph{Decorator} possiamo solo aggiungere
caratteristiche dinamicamente ad un oggetto, mentre nel primo possiamo anche modificarle.
D'altro canto in \emph{Strategy} quando modifichiamo la \emph{Concrete Strategy} di un oggetto
si và a modificare l'oggetto stesso, che è a conoscenza anche delle funzionalità
che gli sono state aggiunte, e non sempre questa è una cosa desiderata. 