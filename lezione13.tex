\subsubsection{Factories}

Una \textbf{\textcolor{cyan}{Factory}} è una classe il cui compito
è quello di creare e restituire istanze di altre classi.

In questo modo viene nascosta, e resa più semplice, la costruzione degli oggetti,
senza dover invocare ogni volta il proprio \emph{costruttore}.

Infatti, ogni volta che viene utilizzato il comando \verb|new|, viene violato il
principio di progettazione di scrivere codice basandosi esclusivamente sulle \emph{interfacce}, 
dato che possiamo associare come istanza di una classe una sua sottoclasse (per il \emph{Principio di Sostituzione di Liskov}).

Inoltre, se abbiamo un frammento di codice che restituisce un'istanza di classe
diversa in base al valore di determinate variabili, allora potrebbero venire violati
i principi di \emph{Open-Closed} e di \emph{Information Hiding}, dato che ogni volta che si
aggiunge o rimuove una classe, occorre anche modificare questo pezzo di codice.

Di seguito vengono trattate le tre tipologie di pattern \emph{\textcolor{cyan}{factories}}.

\paragraph{\textcolor{cyan}{Simple Factory}} Quando si vuole implementare
una logica di creazione di un oggetto molto complessa e la si vuole separare dalle
sue funzionalità, allora si delega la creazione a un oggetto chiamato \textbf{\textcolor{cyan}{factory}}.

Quindi, praticamente, viene spostata la logica di creazione dell'oggetto in un metodo della classe
\emph{Factory}, e in questo modo occorrerà solo invocare questo metodo all'esterno per ottenere un'istanza.
Così se dovesse cambiare il modo in cui creaiamo gli oggetti, non servirà apportare modifiche all'esterno, ma solo
nella \emph{Factory}.

\paragraph{\textcolor{cyan}{Factory Method}} Questo pattern si focalizza sull'uso dell'ereditarietà
per decidere quale oggetto deve essere istanziato.

I partecipanti sono:
\begin{itemize}
    \item \textcolor{cyan}{Product}: definisce l'interfaccia per il tipo
        di oggetti che il \emph{factory method} crea.
    \item \textcolor{cyan}{Concrete Product}: implementa l'interfaccia \emph{Product}.
    \item \textcolor{cyan}{Creator}: dichiara il \emph{factory method} e restituisce un oggetto
        di tipo \emph{Product}.
    \item \textcolor{cyan}{Concrete Creator}: sovrascrive il \emph{factory method} per restituire un'istanza
        di \emph{Concrete Product}.
\end{itemize}

Si usa questo pattern solo se una classe non può prevedere la classe di oggetti che deve creare,
o se una classe richiede delle sottoclassi per specificare gli oggetti da creare.

Ovviamente i benefici solo la grande flessibilità e la riusabilità di questo pattern, ed inoltre dall'esterno
è necessario interfacciarsi solo con la classe \emph{Product}. Gli svantaggi sono che ogni volta che vogliamo
istanziare uno specifico \emph{Concrete Product} occorre creare una sottoclasse di \emph{Creator}. Il \emph{Creator} inoltre
può essere sia astratto che concreto (nel caso si voglia implementare un \emph{factory method}) di default.

\paragraph{\textcolor{cyan}{Abstract Factory}} È una generalizzazione del pattern \emph{Simple Factory}, e la differenza principale col
\emph{Factory Method}, è che mentre lì si utilizza l'ereditarietà basandosi su una sottoclasse che gestisce l'istanzazione di un oggetto,
nell'\emph{Abstract Factory}, una classe delega la responsabilità di creazione di un oggetto a un altro oggetto tramite
la composizione. In realtà, l'oggetto delegato per la creazione, potrebbe utilizzare dei \emph{factory method} per eseguire l'istanzazione, quindi
applicando entrambi i pattern. \\

I \textbf{\textcolor{cyan}{Pure Fabrication}} sono un pattern \emph{GRASP} che si pone l'obbiettivo
di non violare l'\emph{alta coesione} e il \emph{basso accoppiamento}. Questo viene realizzato
assegnando un insieme di responsabilità molto coese fra loro ad una classe artificiale che non rappresenta
niente nel dominio del problema, di conseguenza avremo implementato che il \emph{basso accoppiamento} che favorisce
il riutilizzo.
La progettazione di oggetti può essere partizionata in due gruppi, quelli decomposti sulla rappresentazione, e quelli decomposti
sul comportamento. L'ultimo gruppo non rappresenta proprio niente nel dominio del problema e sono solo costruiti per convenienza
del programmatore, da qui il nome \emph{Pure Fabrication}.

Una \emph{\textcolor{cyan}{Factory}} è un \emph{\textcolor{cyan}{Pure Fabrication}} che ha l'obbiettivo di confinare ed incapsulare
la logica di creazione, soprattutto quand'è abbastanza complessa.

\subsubsection{Singleton}

Questo pattern è utile quando si vuole che un metodo abbia una sola istanza e si vuole
fornire un punto di accesso globale ad essa. Per prevenire le multiple istanze, si crea un'istanza
della classe all'interno della stessa con accesso privato e statica; successivamente si realizza un metodo
pubblico che permette di rendere disponibile l'istanza.

Inoltre si preferisce usare una \emph{Lazy Initialization} dell'istanza, ovvero quando il metodo
pubblico è invocato, se l'istanza non è stata creata si crea e si restituisce, altrimenti si restituisce solo.
Infatti la creazione dell'istanza potrebbe richiedere dei parametri oppure semplicemente è troppo pesante e non avrebbe senso
crearla prima che venga usata o se non verrebbe usata proprio. \\

Il problema si crea quando si lavora in un ambiente \emph{multi-thread}
per colpa delle \emph{race condition}. Una soluzione potrebbe essere quella di sincronizzare
il metodo che restituisce l'istanza, ma sarebbe troppo dispendioso dato che la creazione
dell'istanza avviene solo una volta. In alternativa si può usare un \verb|double-checked-locking|.

Consiste nel dichiarare l'istanza di tipo \verb|volatile| ovvero viene garantito che quando leggiamo il valore
da questa variabile si legge sempre l'ultimo scritto. In seguito nel metodo che restituisce
l'istanza si fà prima un controllo se l'istanza è diversa da \verb|NULL|, successivamente si crea un blocco
\verb|synchronized| che garantisce la \emph{mutua esclusione} sull'intera classe, e al cui interno prima di istanziare
l'oggetto si fà un altro controllo se l'istanza corrente è \verb|NULL| (dato che il controllo precendente non era mutualmente esclusivo).

Se si vogliono creare delle sottoclassi utilizzando \emph{Singleton}, quindi mantenendo sempre una sola istanza, si può fare
in due modi, o nel metodo della superclasse determinare il tipo dell'istanza da creare attraverso un paramtro passato come argomento, anche se in questo
caso non è garantito che i costruttori delle sottoclassi siano privati e in questo caso si potranno istanziare più oggetti delle sottoclassi (inoltre viene anche violato il principio di \emph{Open-Closed} come visto prima);
oppure il modo migliore è quello di spostare il metodo che restituisce l'istanza in ogni sottoclasse.

Tuttavia i vantaggi di usare \emph{Singleton} piuttosto che una \emph{classe statica}, sono
la possibilità di passare l'istanza unica come parametro in un altro metodo, e di poter utilizzare
i \emph{factory} pattern per costruire l'istanza.