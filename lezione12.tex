\subsubsection{State}

È un pattern comportamentale, che consente ad un oggetto di modificare
il suo comportamento quando il suo stato interno cambia.

I partecipanti sono:
\begin{itemize}
    \item \textcolor{cyan}{Context}: rappresenta l'interfaccia che utilizza l'utente e 
        mantiene un'istanza di \emph{Concrete State} che rappresenta lo stato corrente.
    \item \textcolor{cyan}{State}: definisce l'interfaccia degli stati; può anche essere
        una classe concreta o astratta nel caso ci sono comportamenti comuni che non dipendono dallo stato,
        o nel caso si vogliono definire comportamenti di \emph{default}.
    \item \textcolor{cyan}{Concrete State}: è una sottoclasse di \emph{State} che implementa un
        singolo stato. Generalmente sono i \emph{Concrete State} ad effettuare la transizione di stato,
        dato che conoscono il loro \emph{next state}, ma nei casi più semplici si può cambiare stato anche nel
        \emph{Context}.
\end{itemize}