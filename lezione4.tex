\newpage

\section{Linguaggio UML e Casi d'Uso}

\begin{definition}[Modello]
    Un \emph{modello} è un'astrazione del dominio, usato per specificarne la natura e il comportamento.
\end{definition}

I modelli possono classificarsi in:
\begin{itemize}
    \item \textbf{\textcolor{cyan}{Modelli Statici}}: vengono rappresentate le \emph{entità} e le \emph{relazioni} fra esse per permettere di descrivere al meglio
        il dominio, le componenti architetturali e le classi da realizzare.
    \item \textbf{\textcolor{cyan}{Modelli Dinamici}}: vengono modellati i comportamenti delle entità descritte nel \emph{modello statico}.
\end{itemize}

Un modello può essere:
\begin{itemize}
    \item Una \textcolor{cyan}{bozza} o \textcolor{cyan}{sketch}, quindi un modello molto incompleto,
        usato principalmente per descrizioni iniziali.
    \item Un progetto dettagliato chiamato \textcolor{cyan}{blueprint},
        che permette ai programmatori di realizzare direttamente il software senza prendere decisioni di
        progettazione.
    \item Un \textcolor{cyan}{eseguibile}, talmente preciso e completo da poter generare il codice
        in automatico partendo solo dal modello.
\end{itemize}

\subsection{UML}

\begin{definition}[UML]
    L'\emph{Unified Modeling Language} è un linguaggio di modellazione unificato che ha il compito
    di supportare la descrizione e il progetto di software, nello specifico di applicazioni \emph{object oriented}, ma permette
    anche di descrivere i modelli da più punti di vista in modo molto comprensibile sia dai clienti che dagli utenti.
\end{definition}

\subsection{Diagramma dei Casi d'Uso}
Permette di descrivere i \emph{\textcolor{cyan}{requisiti funzionali}} del sistema, catturando nello specifico
le funzionalità viste dall'esterno (lato utente).

\begin{definition}[Attore]
    Un \textbf{\textcolor{cyan}{attore}} è un'entità esterna al sistema, che interagisce con esso.
    Gli attori possono essere classificati in:
    \begin{itemize}
        \item Un \textcolor{cyan}{utente umano} che possiede un determinato ruolo.
        \item Un altro \textcolor{cyan}{sistema}.
        \item Il \textcolor{cyan}{tempo}.
    \end{itemize}
    All'interno del diagramma gli attori sono delle classi e sono indicati con un nome in maiuscolo.
\end{definition}

\begin{definition}[Caso d'Uso]
    Un \textbf{\textcolor{cyan}{caso d'uso}} è una funzionalità o un servizio offerto dal sistema a uno o più attori, e viene
    espresso tramite un insieme di \emph{scenari}.

    All'interno del diagramma, anche i casi d'uso sono scritti in maiuscolo, e per descriverli vengono usati dei \emph{verbi} che ne indicano il compito.
\end{definition}

Il \emph{\textcolor{cyan}{diagramma dei casi d'uso}} oltre ad essere composto da
\emph{attori} e da \emph{casi d'uso}, presenta anche:
\begin{itemize}
    \item \textcolor{cyan}{Relazioni}: tra gli attori e i casi d'uso che rappresentano un'interazione.
    \item Il \textcolor{cyan}{confine del sistema}: un rettangolo disegnato intorno ai casi d'uso per indicare il confine del sistema.
\end{itemize}

È importante specificare che un caso d'uso è sempre iniziato da un solo attore, chiamato \textcolor{cyan}{attore principale}. Inoltre, possono
essere presenti casi d'uso non collegati ad alcun attore.

\begin{figure}[h]
    \centering
\end{figure}

\subsection{Narrativa dei Casi d'Uso}
Per poter descrivere il \emph{modello dinamico}, viene redatto un documento che permette di
rappresentare gli scenari di ogni caso d'uso dal punto di vista di ogni attore coinvolto.

\begin{definition}[Precondizioni e postcondizioni]
    Le \textbf{\textcolor{cyan}{precondizioni}} e le \textbf{\textcolor{cyan}{postcondizioni}} sono
    dei predicati che devono sempre essere veri in uno stato: per le \emph{precondizioni} prima di iniziare il caso d'uso,
    per le \emph{postcondizioni} alla fine.
\end{definition}

La descrizione di un caso d'uso segue questa struttura:

\begin{center}
    \begin{tabular}{||c||}
        \hline
        \emph{Nome} \\
        \hline
        \emph{Breve descrizione} \\
        \hline
        \emph{Attore primario} \\
        \hline
        \emph{Attori secondari} \\
        \hline
        \emph{Precondizioni} \\
        \hline
        \emph{Sequenza degli eventi principale} \\
        \hline
        \emph{Postcondizioni} \\
        \hline
        \emph{Sequenze alternative degli eventi} \\
        \hline
    \end{tabular}
\end{center}

\begin{definition}[Scenario]
    Uno \textbf{\textcolor{cyan}{scenario}} è un'istanza di un caso d'uso, ovvero una sequenza 
    di interazioni tra il sistema e gli attori che produce un risultato osservabile.

    Gli scenari descritti dalla \emph{sequenza degli eventi principale} sono quelli che portano
    alle \emph{postcondizioni}.
\end{definition}

La \emph{\textcolor{cyan}{Sequenza degli eventi principale}} elenca i passi che compongono il caso d'uso ed ogni passo presenta la
seguente sintassi:
\begin{center}
    \verb|<|\emph{numero}\verb|>|. \verb|<|\emph{soggetto}\verb|><|\emph{azione}\verb|><|\emph{complementi}\verb|>|
\end{center}
Il primo passo, inoltre, è sempre compiuto dall'\emph{attore principale}.
All'interno della sequenza possono anche presenti \emph{condizioni} e \emph{cicli}, scritti in pseudocodice.

%manca immagine, approfondimento hoare e liskov.

\subsubsection{Inclusione}

L'\emph{\textcolor{cyan}{inclusione}} permette di creare una relazione di dipendenza tra casi d'uso.

È importante non usare la relazione di inclusione per fare decomposizione di un caso d'uso.

\subsubsection{Estensione}

L'\emph{\textcolor{cyan}{estensione}}, a differenza dell'\emph{inclusione}, non è una dipendenza, ma
permette a un caso d'uso di incorporarne opzionalmente un altro.