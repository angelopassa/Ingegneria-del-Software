\section{Analisi dei Requisiti}

\begin{definition}[Analisi dei Requisiti]
    Si intende il processo di studio e analisi delle esigenze del committente e dell'utente per
    giungere alla produzione di un documento che definisce il \emph{dominio} del problema e i \emph{requisiti} del software.
    In alcuni casi si definiscono anche i \emph{casi di test} e il \emph{manuale utente}.
\end{definition}

Prima di passare alla fase vera e propria di \emph{analisi dei requisiti}
occorre seguire una fase preliminare per stabilire la realizzabilità
del progetto software.

\subsection{Studio di Fattibilità}

Si basa principalmente sulla descrizione del software e delle necessità
dell'utente. In seguito vengono svolte due analisi:

\begin{itemize}
    \item \textcolor{cyan}{Analisi di Mercato}: si fà un confronto con il mercato attuale e si stimano i costi di
        produzione e quanto l'investimento può essere remunerativo.
    \item \textcolor{cyan}{Analisi Tecnica}: si studiano tutti gli strumenti per la realizzazione del progetto,
        come i software, le architteture, gli hardware e gli algoritmi. Inoltre si studia come deve essere fatta la prototipazione
        del processo.
\end{itemize}

\subsection{Dominio}

Il \textbf{\textcolor{cyan}{dominio}} è il contesto in cui il software opera. Per definirlo occorre costruire
un \textcolor{cyan}{glossario}, ovvero una collezione di definizioni di termini rilevanti in quel dominio specifico e
che può essere riusato in progetti successivi nello stesso dominio. Inoltre occorre definire un \textcolor{cyan}{modello statico},
quindi come interagiscono fra loro gli elementi del dominio staticamente, e un \textcolor{cyan}{modello dinamico}, ovvero come si comporta il dominio
in base all'avvenire di un determinato evento. Questi due modelli possono essere descritti sia 
tramite l'uso del linguaggio UML \footnote{\url{https://it.wikipedia.org/wiki/Unified_Modeling_Language}}, sia usando la semplice descrizione testuale.

\subsection{Requisiti}

\begin{definition}[Requisito]
    Il requisito è una proprietà che deve essere garantita dal sistema per soddisfare
    una qualsiasi necessità dell'utente.
\end{definition}

I requisiti possono dividersi in due categorie:
\begin{itemize}
    \item \textbf{\textcolor{cyan}{Requisiti funzionali}}: quelli che descrivono le funzionalità e il comportamento del software.
    \item \textbf{\textcolor{cyan}{Requisiti non funzionali}}: descrivono le proprietà del software o del processo di sviluppo. Ad esemprio le caratteristiche di
        efficienza e affidabilità, l'interfaccia, il linguaggio di programmazione e l'ambiente di sviluppo scelti, i vincoli legislativi e i requisiti hardware o di rete.
\end{itemize}

I requisiti possono essere descritti mediante l'uso di diversi linguaggi, formali o meno. In questo caso si vede
la descrizione dei requisiti mediante la produzione di un documento scritto in linguaggio naturale.

\subsection{Documento dei Requisiti}

Questo documento è un contratto tra lo sviluppatore e il committente, che elenca i requisti e i vincoli che il software deve soddisfare, e specifica
anche una \emph{deadline} per la consegna del progetto.

\subsection{Fasi dell'Analisi dei Requisiti}

L'\emph{analisi dei requisiti} viene svolta in cinque passi:
\begin{enumerate}
    \item \textcolor{cyan}{Acquisizione}
    \item \textcolor{cyan}{Elaborazione}
    \item \textcolor{cyan}{Convalida}
    \item \textcolor{cyan}{Negozazione}
    \item \textcolor{cyan}{Gestione}
\end{enumerate}

\subsubsection{Acquisizione}

Il team di analisti incontra i membri dell'organizzazione del committente e si procede con
la raccolta dei requisiti che può avvenire tramite: semplici interviste, questionari, costruzione di
prototipi (anche su carta), studio di documenti o l'osservazione di possibili utenti mentre lavoro.

\subsubsection{Elaborazione}

Viene scritta la prima bozza del \emph{documento dei requisti}, dove quest'ultimi vengono trattati in modo
più approfondito. La struttura del documento deve essere la seguente:

\begin{center}
    \begin{tabular}{||c||}
        \hline
        \emph{Introduzione} \\
        \hline
        \emph{Glossario} \\
        \hline
        \emph{Definizione dei Requisiti Funzionali} \\
        \hline
        \emph{Definizione dei Requisiti Non Funzionali} \\
        \hline
        \emph{Architettura}: la strutturazione del software in sottosistemi. \\
        \hline
        \emph{Specifica dettagliata dei Requisiti Funzionali} \\
        \hline
        \emph{Modelli astratti}: descrivere il sistema in base \\ a ciascun punto di vista. \\
        \hline
        \emph{Evoluzione del sistema}: successivi cambiamenti. \\
        \hline
        \emph{Appendici}: descrizione della piattaforma hardware, database, \\ manuale utente e i piani di test. \\
        \hline
        \emph{Indici}: costruire un lemmario, quindi una lista di termini \\ che puntano ai requisiti che li usano. \\
        \hline
    \end{tabular}
\end{center}

\paragraph{Nota Bene} Nella descrizione dei requisiti occorre sempre usare la forma assertiva.
Esempio:
\begin{center}
    \emph{Il $<$sistema$>$ deve $<$funzionalità$>$/$<$proprietà$>$}
\end{center}

\subsubsection{Convalida}

Nella fase di convalida occorre revisionare il documento per far sì che vengano
evitati i seguenti difetti:

\begin{itemize}
    \item \textcolor{cyan}{Omissioni}: requisiti mancanti.
    \item \textcolor{cyan}{Inconsistenze}: contraddizione tra i requisiti o tra un requisito e il contesto.
    \item \textcolor{cyan}{Ambiguità}: vaghezze o requisiti che possono avere più significati. Le ambiguità all'interno del linguaggio naturale
        possono essere portate da \textcolor{MidnightBlue}{quantificatori}, \textcolor{MidnightBlue}{disgiunzioni}
        oppure possono presentarsi ambiguità di \textcolor{MidnightBlue}{coordinazione} (nel caso si usano sia la \emph{o} che la \emph{e} nella stessa frase) oppure 
        \textcolor{MidnightBlue}{referenziale} nell'uso non chiaro di pronomi.
        Inoltre occorre sempre evitare \textcolor{MidnightBlue}{verbi deboli}, \textcolor{MidnightBlue}{forme passive}, ovvero verbi senza un soggetto esplicito ed anche \textcolor{MidnightBlue}{negazione}
        e \textcolor{MidnightBlue}{doppie negazioni}.
    \item \textcolor{cyan}{Sinonimi} e \textcolor{cyan}{Omonimi}: termini diversi con lo stesso significato e termini uguali con significato diverso.
    \item \textcolor{cyan}{Presenza di dettagli tecnici}
    \item \textcolor{cyan}{Ridondanza}: può esserci, ma solo tra sezioni diverse del documento.
\end{itemize}

\subsubsection{Negozazione}

\subsubsection{Gestione}