\subsection{Criteri White-Box}

Sono criteri che si basano sulla struttura del codice per individuare
i casi di input. Aiutano ad aggiungere altri test oltre a quelli generati
con quelli di tipo \emph{black-box}.

Si dice che un programma non è testato adeguatamente se tutti i suoi cammini
non vengono esercitati dai test.

Un \emph{\textcolor{cyan}{grafo di flusso}} definisce la struttura
del codice identificandone tutte le parti collegate fra loro.

Quindi si cerca un insieme di valori per i dati di input in modo tale che
vengano esercitati tutti i comandi e tutte le condizioni del programma.

La \textbf{\textcolor{cyan}{Misura di Copertura}} per i \textbf{comandi} è definita come segue:
\begin{equation*}
    Misura\;di\;Copertura = \frac{Numero\;di\;Comandi\;Esercitati}{Numero\;di\;Comandi\;Totali}
\end{equation*}

È importante sottolineare che la \emph{copertura} non è monotona rispetto
alla dimensione dell'insieme di test. Infatti ci possono essere insiemi più piccoli
che magari garantiscono una copertura maggiore.

Se in un caso di test vengono eseguiti tutti comandi, non è detto che vengano
eseguite anche tutte le condizioni, quindi si parla anche di \textbf{\textcolor{cyan}{Misura di Copertura}}
per le \textbf{condizioni}:
\begin{equation*}
    Misura\;di\;Copertura = \frac{Numero\;di\;Archi\;Entranti}{Numero\;di\;Archi\;Totali}
\end{equation*}

Per le \emph{condizioni composte}, bisogna anche fare in modo che i test
esercitino tutti i possibili valori di verità delle singole condizioni semplici all'interno.

\paragraph{\textcolor{cyan}{Copertura di Condizioni Semplici}} Un insieme di test
per un programma copre tutte le condizioni semplici (\emph{\textcolor{cyan}{basic condition}})
se per ogni condizione semplice nel programma, l'insieme di test ne contiene almeno
uno in cui la condizione vale \verb|true| e almeno uno in cui vale \verb|false|.

\begin{equation*}
    Copertura\;delle\;Basic\;Condition = \frac{Numero\;di\;Valori\;di\;Verit\grave{a}\;Assunti}{2^{Numero\;di\;Basic\;Condition}}
\end{equation*}

Nella \emph{Copertura delle Multiple Condition}, per ogni condizione multipla con $n$
condizioni semplici, invece, si cerca di coprire tutte le $2^n$ possibili combinazioni.
Ma grazie alla valutazione \emph{short-circuit} \footnote{\url{https://en.wikipedia.org/wiki/Short-circuit_evaluation}}
di molti linguaggi di programmazione ci si può ridurre a meno.

La \emph{Copertura dei Cammini}, invece richiede di percorrere tutti i cammini, cresce in modo
esponenziale col il numero di decisioni. Ovviamente in presenza di cicli
il numero di cammini è potenzialmente infinito, per questo ci si limita a
richiedere casi di test che esercitino ogni ciclo 0 volte, 1 volta e più di una volta.

\subsection{Fault Based Testing}

Ipotizza dei difetti potenziali nel codice sotto test, ovvero si valuta
un insieme di test sulla base della loro capacità di rilevare i difetti ipotizzati.

La tecnica più nota è il \textbf{\textcolor{cyan}{Test Mutazionale}}, ovvero
si iniettano difetti modificando il codice.

Occorre priva aver esercitato un programma $P$ su una batteria di test $T$
e si verifica che $P$ sia corretto rispetto a $T$. Successivamente si introducono
dei piccoli difetti in $P$ chiamate \emph{mutazioni}. Il programma \emph{mutante} è
indicato con $P'$. Quindi si eseguono su $P'$ gli stessi test di $T$.

Se il test non rileva i difetti iniettati, allora significa che la batteria di test non
era buona, se li rileva si ha una maggiore fiducia su essa.

Nella realtà si applicano tante piccole mutazioni a $P$ per ottenere una
sequenza di mutanti $P_1, P_2, \dots, P_n$. Si dice che $T$ uccide il mutante
$P_j$ se rileva un malfunzionamento in almeno un caso di test di $T$.

L'\textbf{\textcolor{cyan}{Efficacia di un Test}} è il rapporto $\frac{Nr.\;Mutanti\;Uccisi}{Numero\;Totale\;Mutanti}$.

Un mutante sopravvive a una test suite se per tutti i test case della batteria non si
distingue l'esito del test se eseguito sul programma originale o sul mutante.

Qui si lavora sull'assunzione che iniettando difetti piccoli e artificiali è
possibile individuare difetti più complessi e reali.

Praticamente si iniettano $M$ difetti meccanici nell'originale ottenendo il mutante,
si esegue il mutante sulla batteria, e si troverà un numero di difetti uguale a $N$.
Se $N > M$ allora ho trovato difetti non artificiali che erano presenti anche nel programma originale.

Esempi di mutazioni:
\begin{itemize}
    \item \textcolor{cyan}{CRP}: sostituzione di una costante per una costante.
    \item \textcolor{cyan}{ROR}: sostituzione di un operatore relazione (da $\leq$ a $<$).
    \item \textcolor{cyan}{VIE}: eliminazione dell'inizializzazione di una variabile.
    \item \textcolor{cyan}{LRC}: sostituzione di un operatore logico.
    \item \textcolor{cyan}{ABS}: inserimento di un valore assoluto.
\end{itemize}

Un \emph{mutante} si dice \textbf{\emph{\textcolor{cyan}{invalido}}} se non è
sintatticamente corretto, ovvero se non passa la compilazione, altrimenti si dice
\textbf{\emph{\textcolor{cyan}{valido}}}.

Un \emph{mutante} si dice \textbf{\emph{\textcolor{cyan}{utile}}} se è \emph{valido}
e distinguerlo dal programma originale non è facile, ovvero esiste solo un piccolo sottoinsieme
di casi di test che lo consente. Invece si dice \textbf{\emph{\textcolor{cyan}{inutile}}} se è
ucciso da quasi tutti i casi di test.

Si cerca di costruire mutazioni che producano mutanti \emph{utili} e \emph{validi}.

Un mutante può essere \textbf{\emph{\textcolor{cyan}{equivalente}}} al programma originale
se ad esempio la mutazione non crea un vero difetto o se la batteria di test era troppo debole.

Ovviamente questa tecnica è limitata dal numero di mutanti che possono essere definiti,
dato che il loro costo di realizzazione, il tempo e le risorse necessarie ad eseguire
i test possono essere molto elevati.

\subsection{Oracolo ed Output Attesi}

\begin{definition}[Oracolo]
    Un \textbf{\textcolor{cyan}{oracolo}} è uno strumento utilizzato per generare
    i risultati attesi di una batteria di test. La sua importanza è che fornisce
    un punto di riferimento per la correttezza del sistema durante il testing.
\end{definition}

L'\emph{output atteso} può essere trovato in diversi modi:
\begin{itemize}
    \item \emph{Risultati ricavati dalle specifiche}.
    \item \emph{Inversione delle funzioni}: quando la funzione inversa è più facile. In questo modo
        dall'output si calcola l'input.
    \item \emph{Versioni precedenti del codice}: per le funzionalità non modificate.
    \item \emph{Versioni multiple indipendenti}: programmi preesistenti, oppure semplificazioni
        di algortimi, che sono poco efficienti ma corrette.
    \item \emph{Semplificazione dei dati d'ingresso}: si provano le funzionalità su dati semplici,
        oppure i risultati sono noti tramite l'utilizzo di altri mezzi, o addirittura ci si mette
        nell'ipotesi di avere un comportamento costante.
    \item \emph{Semplificazione dei risultati}: ci si basa su risultati plausibili,
        ma si applicano dei vincoli fra gli input e gli output, e delle invarianti
        sugli output.
\end{itemize}

\subsection{Test di Sistema}

\begin{itemize}
    \item \textcolor{cyan}{Facility Test}: mira a controllare che ogni funzionalità stabilita nei requisiti
        sia stata realizzata correttamente.
    \item \textcolor{cyan}{Security Test}: si cerca di accedere a dati e funzionalità che dovrebbero essere riservate.
    \item \textcolor{cyan}{Usability Test}: si valuta la facilità d'uso del software da parte dell'utente finale. Ci si basa
        sulla documentazione e sul livello di competenza dell'utenza.
    \item \textcolor{cyan}{Performance Test}: si valuta l'efficienza di un sistema rispetto ai tempi di elaborazione
        e risposta. Si sottopone il sistema a diversi livelli di carico.
    \item \textcolor{cyan}{Volume Test}: qui si sottopone il sistema al carico di lavoro massimo stabilito
        nei requisiti. Si cercano malfunzionamenti che non si presentano normalmente.
    \item \textcolor{cyan}{Stress Test}: si sottopone il sistema a carichi di lavoro superiori a quelli previsti
        nei requisiti, oppure si porta in condizioni non previste, ad esempio sottraendogli risorse
        di memoria e calcolo. Lo scopo è quello di controllare la capacità di \emph{recovery}
        del sistema dopo un fallimento.
    \item \textcolor{cyan}{Storage Use Test}: è un controllo mirato alla richiesta delle risorse, in particolare
        la memoria, ed ha implicazioni sulla definizione dei requisiti minimi del sistema per poter installare il software.
    \item \textcolor{cyan}{Configuration Test}: si prova il sistema su sistemi operativi diversi e con
        differenti hardware installati.
    \item \textcolor{cyan}{Compatibility Test}: si valuta la compatibilità del software con altri, anche
        con una sua versione precendente che deve rimpiazzare.
\end{itemize}
